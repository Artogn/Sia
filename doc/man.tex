\documentclass[a4paper,10pt]{article}
%\documentclass[a4paper,10pt]{scrartcl}

\usepackage[utf8]{inputenc}
\usepackage{color}
\usepackage[T1]{fontenc}
\usepackage{pstricks}
\usepackage{pst-tree}
\usepackage{graphicx}
\usepackage{pst-node}
\definecolor{WALLET}{RGB}{0,255,0}
\title{Developer's Manual}
\author{Alexander White}
\date{2/20/2014}


\begin{document}
\maketitle
\tableofcontents
\section{Overview}
\paragraph{} \raggedright Sia is a new cryptocurrency that uses 
proof-of-storage instead of proof-of-work. This creates a blockchain that is 
protected by an expensive resource that is also useful for storing files over a 
distributed network.
\paragraph{}The proof-of-storage algorithm uses a blocktree instead of a 
blockchain. This means that each participant in the network only sees the 
transactions that are relevant to them, instead of seeing every transaction 
that occurs over the network. This is done in a secure and trust-free way.
\paragraph{}Finally, Sia introduces the concept of script-based wallets as 
opposed to public key based wallets. This adds an enormous amount of power to 
wallets by allowing them to have arbitrary volumes of public keys and arbitrary 
rules for committing transactions.\linebreak
\section{The Sia Blocktree}
\paragraph{}Each node in the blocktree is a blockchain. Each blockchain tracks 
a random subset of the wallets on the network, and each wallet belongs to only 
a single blockchain. Each parent node has aggregate information about it’s 
children, as in the following example:\linebreak
\resizebox{\linewidth}{!}{
\psframebox[fillcolor=gray,fillstyle=solid]{%
    \begin{psTree}{\Tcircle[treefit=tight,levelsep=5pt, nodesep=3pt]{Root}}
	\begin{psTree}{\Tcircle{1}}
	    \begin{psTree}{\Tcircle{1.1}}
		\Toval[fillcolor=WALLET]{\small 1.1.W1}\tlput{Bal:3}
		\Toval[fillcolor=WALLET]{\small 1.1.W2}\trput{Bal:1}
	    \end{psTree}
	    \begin{psTree}{\Tcircle{1.2}}
		\Toval{\small 1.2.W1}\trput{Bal:6}
		\Toval{\small 1.2.W2}\trput{Bal:4}
	    \end{psTree}
	\end{psTree}
	\begin{psTree}{\Tcircle{2}}
	    \begin{psTree}{\Tcircle{2.1}}
		\Toval[fillcolor=WALLET]{\small 2.1.W1}\tlput{Bal:1.3}
		\Toval[fillcolor=WALLET]{\small 2.1.W2}\trput{Bal:6}
	    \end{psTree}
	    \begin{psTree}{\Tcircle{2.2}}
		\Toval[fillcolor=WALLET]{\small 2.2.W1}\tlput{Bal:1}
		\Toval[fillcolor=WALLET]{\small2.2.W2}\trput{Bal:5}
	    \end{psTree}
	    
	\end{psTree}
	

    \end{psTree}
}}
\paragraph{}
\indent
In the diagram you can see that blockchain 1.1 tracks wallets 1.1.w1 and 
1.1.w2. The combined balances of all the wallets in 1.1 is 4, so 1.1 reports 
its aggregate balance as 4. Similarly, blockchain 1 tracks the two blockchains 
1.1 and 1.2, which have a combined balance of 14. Blockchain 1 then reports its 
aggregate balance as 14.
\paragraph{}
When sending a transaction, transactions travel through the tree:
\linebreak
\hbox{

  \begin{psTree}{\Tcircle[name=root]{root}}
  \begin{psTree}{\Tcircle[name=one]{1}}
  \Tr[name=oneone]{1}
  \Tr[name=onetwo]{2}
  \end{psTree}
  \begin{psTree}{\Tcircle[name=two]{2}}
    \Tr[name=twotwo]{2}
    \Tr[name=twoone]{1}
  \end{psTree}
  \end{psTree}
  \ncline[offset=6pt]{->}{oneone}{one}\ncput*[nrot=:U]{\tiny 2 Coins}
  \ncline[offset=6pt]{->}{one}{root}\ncput*[nrot=:U]{\tiny 2 coins}
  \ncline[offset=-6pt]{->}{root}{two}\ncput*[nrot=:U]{\tiny 2 coins}
  \ncline[offset=-7pt]{->}{twoone}{two}\nbput*[nrot=:D]{\tiny 3 Coins}
  \ncline[offset=-5pt]{->}{two}{twotwo}\nbput*[nrot=:D]{\tiny 2 coins}
  \ncline[offset=5pt]{->}{two}{twotwo}\ncput*[nrot=:D]{\tiny 3 coins}


}

%  }

\paragraph{}
Here we have two example transactions. The first is from 2.w1 to 2.w2. 
Blockchain 2 confirms that 2.w1 has sufficient balance, and then confirms the 
transaction. In this case, only a portion of the network knows about the 
transaction, as the aggregate balance of blockchain 2 did not change.
\paragraph{}
In the second example, 1.w1 sends money to 2.w2. First blockchain 1 confirms 
that 1.w1 has sufficient balance. Then, the root blockchain confirms that 
blockchain 1 has enough aggregate coins to send the transaction to 2.w2. The 
root blockchain does not know about wallet 1.w1, it only knows that blockchain 
1 has a large enough aggregate balance to send money to 2.w2. Finally, 
blockchain 2 receives the transaction and allocates the coin to wallet 2.w2.

\section{Individual Blockchains and Blocks}
\paragraph{}
Each blockchain in the blocktree is composed of n hosts (n being the same for 
every blockchain in the tree). Each host contributes the same volume of 
persistent storage to the network, this makes becoming a host difficult (and 
therefore prevents sybil attacks). (A single machine can participate as 
multiple hosts provided the machine has enough storage).
\paragraph{}
	In each block, certain tasks must be accomplished:
\begin{enumerate}
\item Produce entropy for the local blockchain (to seed random numbers)
\item Each host must prove they still fulfill the persistent storage 
requirement
\item All transactions must be processed
\end{enumerate}
\par
	This is accomplished using a consensus algorithm. In the first step, 
every host sends its own update to every other host. In the second step, every 
host receives an acknowledge. When a host sends an acknowledge, the claim is 
‘if my update appears in the blockchain, your update will also appear in the 
blockchain.’
\par
	In the second step, each host contacts a block compiler. There is 1 
block compiler each block, and the block compiler is randomly chosen from the n 
hosts. The block compiler is actually an ordering, so if one block compiler is 
offline the next is already known.
\par
	Each host sends every update that they acknowledged (including their 
own) to the block compiler, who then returns the acknowledge. If the block 
compiler fails to return an acknowledge, the next block compiler is contacted 
until an acknowledge is sent.
\par
Once contacted by a majority of the network (or after a significant amount of 
time has expired), the block compiler announces the block to the blockchain. 
The block that is accepted is the block announced by the block compiler with 
updates from the most hosts.
\par
If the majority of hosts are honest, then every single honest host will have 
their update included in the block, because each honest host will have an 
update from each honest host, and will therefore keep contacting block 
compilers until one of them acknowledges the stack of honest updates. A block 
cannot have the majority of updates without also having all of the honest 
updates, so no dishonest block will be accepted.
\par
Each block is sent to all children blockchains, which allows the children to 
verify the authenticity of the network. This also means that the children will 
be aware of all incoming transactions targeted toward them.
\par
	To send a transaction to a parent, the block signs the transaction and 
announces it to the parent blockchain in the same way that transactions are 
normally announced to a blockchain.
\subsection{entropy}
\par
	Significant portions of the network security rely on randomness. 
Entropy is generated on a per blockchain basis, and it is generated in two 
stages. In the first stage, each host in the blockchain picks a random string 
and shares the hash of that random string. The method of choosing the random 
string is left to the hosts. In the second stage, the hosts reveal the random 
string that matches the hash from the first stage. The strings are then all 
combined in a predetermined way and used as a seed in a cryptographically 
secure random number generator.
\par
	By doing this, if even a single host is generating random numbers, the 
entire blockchain will be seeded to produce random numbers.
\par
	(!) This method is vulnerable to attacks where hosts deliberately go 
offline (or force other hosts offline through DDOS or other methods) in order 
to manipulate the random string generated in the second stage. The current 
defense is to maintain strong punishments for going offline to make entropy 
manipulation not-worthwhile.
\subsection{Proving Storage}
\par
	Each host has an identical volume of persistent storage that they 
contribute to the network. Each host has what is called a ‘stack’ of data. This 
stack is broken into ‘slices.’ Some of the slices are used for storing the 
network state and the blockchain history. The remaining data is being rented 
out by the sia network, and is filled with arbitrary data. This data is erasure 
coded using Reed-Solomon coding, which means that every single host has 
redundant information about the data stored on every single other host.
\par
	Every block, a random 1024 bit segment of the data stack will be 
chosen, and each host will have to reveal what data they have stored for that 
1024 bit segment. Because each segment is erasure coded across every host in 
the blockchain, the erasure coding repair rules can be used to determine who 
has the file.
\par
	Just like for entropy, proving storage has to be done in two stages to 
prevent information from leaking to dishonest hosts during the first stage. 
(you can figure out what your segment is supposed to look like if you have some 
fraction of the other segments).
\subsection{Renting Storage}
\par
	The sia blocktree requires hosts to contribute large volumes of storage 
to the network, which means that the network has large amounts of storage 
available for use by the public.
\par
This storage can be rented by the public. The price is set in siacoins (scn), 
and fluctuates according to supply and demand. If there is lots of unrented 
storage, the price will drop. If storage is being rented as fast as it is being 
supplied, the price will increase. The money spent by the public on storage 
will go to the hosts providing the storage.
\par
	Because the network has a valuable resource (persistent storage) that 
can only be purchased using scn, scn is given a strong base value. Should the 
value of the scn drop significantly, storage becomes very cheap and demand for 
the storage should increase. Should the value of the scn raise significantly, 
the supply of storage should increase as there is a greater incentive to 
provide storage to the network. The greater value of storage on the network 
will protect the increased price of the scn.
\subsection{Sia Wallets}
	Each wallet is owned by a single blockchain. The data for a wallet is 
stored using the persistent storage in the blockchain. Each node keeps all 
wallet data for the blockchain, meaning wallet data is stored with perfect 
redundancy.
\par
	Because wallets consume storage on the network, they are charged for 
the storage they consume. When a wallet no longer has enough balance to pay for 
the storage it is consuming, the wallet is deleted from the network. This 
allows wallets to be arbitrarily large, as large wallets will be expensive and 
will be deleted if they can no longer rent themselves.
\par
	Wallets have a balance and a script, as opposed to a public key. The 
script determines what transactions go through the network. The trivial case is 
a script that only allows transactions through if they are signed by a specific 
public key. More advanced cases can require sets of public keys, require prices 
to be in certain ranges, or use databases to keep a dynamic program and 
intricate set of rules for allowing transactions to go through.
\hbox{\resizebox{!}{5cm}{
\begin{pspicture}
\psTree[nodesep=1pt]{\Toval{Wallet}}
\Toval{\tiny Balance:256 bits}
\skiplevel{
\Toval{\tiny \label{script} Scripts(arbitrarily large)}
\Toval{\tiny Data(Arbitrarily large,for use by scripts}
}
\endpsTree
\end{pspicture}
}}
\subsection{Sia Scripts}
\par
	A script is a set of code that is run on blockchains. Scripts are 
tethered to wallets and are classified into two categories. The first category, 
scheduled scripts, are scripts that can be be run at specific times or at 
regular intervals. The second category, triggered scripts, are run any time 
someone makes a transaction out of the wallet.\footnote{That being said, this 
will be a \emph{lisp} interpreter}
\par
The scripts are run using a set of defined operations called the sia bytecode. 
Though not yet determined, this script is intended to be turing complete, and 
each operation is to have a cost. This gives scripts a price based on 
computational complexity, and allows the network to price scripts based on 
supply and demand for computational power. Wallets will be charged after each 
operation, and the script will stop running if a wallet runs out of money.
\par
When a script runs, it will either terminate because the code finishes running, 
because it hits a preset limit for maximum number of operations, or because the 
balance paying for the script to run has been emptied while paying for the 
script.
\par
Sia intends to be the most efficient and best platform for cloud storage in the 
world. It does not share these ambitions with the scripting system. The 
scripting system is meant to be powerful but is not intended to be a general 
platform for distributed computing. Sia scripts are likely to be expensive and 
slow.
\subsection{Mining}
Every day, 1000 scn are injected into the network. They are distributed evenly 
among the hosts in the blocktree. The amount of scn being injected into the 
network per day does not change.
\par
The network also starts with 20,000 premined coins. These coins will be 
used by the developers to help establish the network.
\par
There is also a possibility of a permanently inflationary currency, set 
between 1% and 5% per year. This causes coins to lose value with time, and 
distributes the wealth of the network to the people currently contributing, as 
opposed to leaving a large percentage of the money with early adopters.
\subsection{Joining the Network}
\par
	When a host joins the network, they must first wait for a blockchain to 
be created for them to join. Blockchains are created based entirely on demand 
for storage - if nobody is renting storage then the network will have very few 
blockchains regardless of how many hosts wish to join the network.
\par
When there is space on the network for more hosts, new hosts will be 
placed in random existing blockchains throughout the network, displacing hosts. 
This is because an attacker may have an easier time controlling the pool of new 
hosts. The hosts that get displaced will be used in the new blockchains. This 
makes sure that each blockchain is composed of random hosts, regardless of when 
the blockchain was created or when the hosts joined the network. This 
randomness is vital to protecting individual blockchains against weak attackers.
\subsection{Indictments}
\par
Every action performed by a host on the network is signed. If a host 
signs a transaction that they should not have signed, the transaction plus the 
signature can be used to indict the host. Honest hosts will recognize the 
dishonesty of the other host, and that host will be ejected from its blockchain.
\subsection{Dropping Hosts}
\par
Any time that a host fails to keep up with network duties, that host is 
dropped from its blockchain. An attacker could DDOS honest hosts repeatedly 
until a particular blockchain is no longer random, but filled with 
attacker-controlled hosts. This happens because only honest hosts are being 
dropped.
\par
To prevent this attack vector, each time a host is dropped from a 
blockchain two additional hosts are chosen randomly to also be dropped from the 
blockchain (or rather, rotated out to a random other blockchain). This prevents 
the attacker from gaining a statistical advantage because as honest hosts drop, 
attacker hosts will also be rotated out and the attacker will be prevented from 
owning a majority of a blockchain.


\end{document}
